\documentclass[a4paper]{article}

\usepackage[a4paper,vmargin={20mm,20mm},hmargin={20mm,20mm}]{geometry}

\usepackage[pdftex]{graphicx}

\usepackage{amssymb, amsmath, amsthm}

\usepackage{enumitem}

\newcommand {\C} [1] {{\mathbb C}^{#1}}

\newcommand {\R} [1] {{\mathbb R}^{#1}}

\newcommand {\limit} [2] {\displaystyle{\lim_{{#1}\rightarrow{#2}}}}

\newcommand {\bfrac} [2] {\displaystyle{\frac{#1}{#2}}}

\newcommand {\real} {\mbox{Re}}

\newcommand {\imag} {\mbox{Im}}

\newcommand{\br} [1] {\overline{#1}}

\newcommand{\tab} {\hspace{5mm}}

\newcommand{\mmod} [3] {{#1} \equiv {#2} \hspace{1mm} (\bmod{\hspace{1mm}#3})}

\newcommand{\nmod} [3] {{#1} \not\equiv {#2} \hspace{1mm} (\bmod{\hspace{1mm}#3})}

\newcommand{\intm} [1] {\mathbb{Z}_{#1}}

\newcommand {\Z} {\mathbb{Z}}

\newcommand {\intg} [4] {\int_{#1}^{#2} #3 d#4}

\newtheorem{theorem}{Theorem}[section]

\newtheorem{lemma}[theorem]{Lemma}

\newtheorem{cor}[theorem]{Corollary}

\newtheorem{prop}[theorem]{Proposition}

\newtheorem{definition}[theorem]{Definition}

\newtheorem{remark}[theorem]{Remark}

\newtheorem{example}[theorem]{Example}

\numberwithin{equation}{section}

\begin{document}

\begin{flushright}
{\small{Nathan Sponberg}}
\end{flushright}

\begin{center}
\bf{Homework 5}
\end{center}


\begin{description}

\item{\bf{Exercise 1.38}}\\

\item We will calculate the Fourier series for the function 

$$\cos^{2N}(x)\,.$$

Using the exponential notation for $\cos(x)$ we obtain

$$\cos^{2N}(x) = \frac{(e^{ix}+e^{-ix})^{2N}}{2^{2N}}\,.$$

Then, by the binomial theorem, we can expand the numerator as follows

$$\frac{(e^{ix}+e^{-ix})^{2N}}{2^{2N}} = \frac{\sum_{k=0}^{2N}{2N \choose k}e^{ix(2N-k)}e^{-ixk}}{2^{2N}} = \frac{1}{2^{2N}}\sum_{k=0}^{2N}{2N \choose k}e^{ix(2N-2k)}\,.$$

We note that this final expression is a Fourier series. Thus, we are done.

\item{\bf{Exercise 1.41}}\\

\item We will calculate the Fourier series for the function 

\begin{displaymath}
   f(x) = \left\{
     \begin{array}{ll}
       -1 & : -\pi < x < 0\\
       1 & : 0 < x < \pi\\
     \end{array}
   \right.
\end{displaymath} 

First, we calculate the Fourier coefficients. Note that since $f(x)$ is an odd function

$$\hat{f}(0) = \frac{1}{2\pi}\intg{-\pi}{\pi}{f(x)}{x} = 0\,.$$

Now consider the case for which $n \neq 0$. We have

$$\hat{f}(n) = \frac{1}{2\pi}\intg{-\pi}{\pi}{f(x)e^{-inx}}{x} = \frac{1}{2\pi}\Big[\intg{-\pi}{\pi}{f(x)\cos(nx)}{x} + \intg{-\pi}{\pi}{f(x)\sin(nx)}{x}\Big]\,.$$

Then since $f(x)$ is odd and $\cos(x)$ is even, the left integral must be equal to zero. Thus, we have

$$\frac{1}{2\pi}\Big[\intg{-\pi}{\pi}{f(x)\cos(nx)}{x} + \intg{-\pi}{\pi}{f(x)\sin(nx)}{x}\Big] = \frac{1}{2\pi}\intg{-\pi}{\pi}{f(x)\sin(nx)}{x} = $$

$$\frac{1}{\pi}\intg{0}{\pi}{f(x)\sin(nx)}{x} = \frac{i}{\pi}\frac{-\cos(nx)}{n}\Big|_0^\pi = \frac{i(-1)^{n+1} + i}{n\pi}\,.$$

Observe that this final expression is equal to zero for $n = 2k$ and is equal to $2i/n\pi$ for $n = 2k+1$. Thus, the Fourier series for $f(x)$ is

$$\sum_{k<0} \frac{2i}{(2k+1)\pi}e^{i(2k+1)x} + \sum_{k>0} \frac{2i}{(2k+1)\pi}e^{i(2k+1)x} = \sum_{k>0} \frac{2i}{-(2k+1)\pi}e^{-i(2k+1)x} + \sum_{k>0} \frac{2i}{(2k+1)\pi}e^{i(2k+1)x} = $$

$$\sum_{k>0}\frac{2i}{(2k+1)\pi}e^{i(2k+1)x} - \frac{2i}{(2k+1)\pi}e^{-i(2k+1)x} = \sum_{k>0}\frac{2i}{(2k+1)\pi}(e^{i(2k+1)x} - e^{-i(2k+1)x}) = $$

$$\sum_{k>0}\frac{2i}{(2k+1)\pi}2i\sin[(2k+1)x] = \sum_{k>0}\frac{-4}{2k\pi+\pi}\sin(2kx+x)\,.$$

\item{\bf{Exercise 1.42}}\\

\item We will calculate the Fourier series for the function 

$$f(x) = e^ax\,.$$

The Fourier coefficients are

$$\hat{f}(0) = \frac{1}{2\pi}\intg{0}{2\pi}{e^{ax}}{x} = \frac{e^{2a\pi} - 1}{2a\pi}\,,$$

and 

$$\hat{f}(n) = \frac{1}{2\pi}\intg{0}{2\pi}{e^{ax}e^{-inx}}{x} = \frac{1}{2\pi}\frac{e^{(a-in)x}}{a-in}\Big|_0^{2\pi} = \frac{1}{2\pi}\frac{e^{(a-in)2\pi}-1}{a-in} = \frac{1}{2\pi}\frac{e^{2a\pi}-1}{a-in}\,.$$

Now we can calculate the Fourier series for $f(x)$. Observe that

$$\sum_{n \in \mathbb{Z}}\hat{f}(n)e^{inx} = \frac{e^{2a\pi}-1}{2\pi}\Bigg[\frac{1}{a} + \sum_{n<0}\frac{e^{inx}}{a-in} + \sum_{n>0}\frac{e^{inx}}{a-in}\Bigg] = $$

$$\frac{e^{2a\pi}-1}{2\pi}\Bigg[\frac{1}{a} + \sum_{n>0}\frac{e^{-inx}}{a+in} + \sum_{n>0}\frac{e^{inx}}{a-in}\Bigg] = \frac{e^{2a\pi}-1}{2\pi}\Bigg[\frac{1}{a} + \sum_{n>0}\frac{(a-in)e^{-inx}+(a+in)e^{inx}}{a^2+n^2}\Bigg] = $$

$$\frac{e^{2a\pi}-1}{2\pi}\Bigg[\frac{1}{a} + \sum_{n>0}\frac{2a\cos(nx)+2ini\sin(nx)}{a^2+n^2}\Bigg] = \frac{e^{2a\pi}-1}{\pi}\Bigg[\frac{1}{2a} + \sum_{n>0}\frac{a\cos(nx)-n\sin(nx)}{a^2+n^2}\Bigg]\,.$$

\item{\bf{Exercise 1.43}}\\

\item We will calculate the Fourier series for the function 

$$f(x) = \sinh(x) = \frac{e^x-e^{-x}}{2}\,.$$

The Fourier coefficients are as follows 

$$\hat{f}(0) = \frac{1}{2\pi}\intg{-\pi}{\pi}{\sinh(x)}{x} = 0$$

and

$$\hat{f}(n) = \frac{1}{2\pi}\intg{-\pi}{\pi}{\frac{e^x-e^{-x}}{2}e^{-inx}}{x} = \frac{1}{2\pi}\intg{-\pi}{\pi}{\frac{e^{(1-in)x}-e^{(-1-in)x}}{2}}{x} = $$

$$\frac{1}{4\pi}\Big[\frac{e^{(1-in)x}}{1-in} + \frac{e^{(-1-in)x}}{1+in}\Big]_{-\pi}^{\pi} = \frac{(-1)^n}{4\pi}\Big[\frac{e^{\pi} - e^{-\pi}}{1-in} + \frac{e^{-\pi}-e^{\pi}}{1+in}\Big] = $$

$$\frac{(-1)^n}{2\pi}\Big[\frac{\sinh(\pi)}{1-in} - \frac{\sinh(\pi)}{1+in}\Big] = \frac{(-1)^n}{2\pi}\Big[\frac{(1+in)\sinh(\pi) - (1-in)\sinh(\pi)}{1+n^2}\Big] = $$

$$\frac{(-1)^n}{\pi}\Big[\frac{in\sinh(\pi)}{1+n^2}\Big]\,.$$

Now we can proceed to calculating the Fourier series. Observe that

$$\sum_{n \in \mathbb{Z}}\hat{f}(n)e^{inx} = \frac{\sinh(\pi)}{\pi}\Big[\sum_{n<0}\frac{(-1)^nin}{1+n^2}e^{inx} + \sum_{n>0}\frac{(-1)^nin}{1+n^2}e^{inx}\Big] = $$

$$\frac{\sinh(\pi)}{\pi}\Big[\sum_{n>0}\frac{(-1)^{n+1}in}{1+n^2}e^{-inx} + \sum_{n>0}\frac{(-1)^nin}{1+n^2}e^{inx}\Big] = \frac{\sinh(\pi)}{\pi}\Big[\sum_{n>0}\frac{(-1)^nin}{1+n^2}e^{inx} - \frac{(-1)^{n}in}{1+n^2}e^{-inx}\Big] = $$

$$\frac{\sinh(\pi)}{\pi}\Big[\sum_{n>0}\frac{(-1)^nin}{1+n^2}(e^{inx} - e^{-inx})\Big] = \frac{\sinh(\pi)}{\pi}\Big[\sum_{n>0}\frac{(-1)^{n+1}2n\sin(nx)}{1+n^2}\Big]\,.$$



\end{description}

\end{document} 