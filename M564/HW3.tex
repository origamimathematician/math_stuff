\documentclass[a4paper]{article}

\usepackage[a4paper,vmargin={20mm,20mm},hmargin={20mm,20mm}]{geometry}

\usepackage[pdftex]{graphicx}

\usepackage{amssymb, amsmath, amsthm}

\usepackage{enumitem}

\newcommand {\C} [1] {{\mathbb C}^{#1}}

\newcommand {\R} [1] {{\mathbb R}^{#1}}

\newcommand {\limit} [2] {\displaystyle{\lim_{{#1}\rightarrow{#2}}}}

\newcommand {\bfrac} [2] {\displaystyle{\frac{#1}{#2}}}

\newcommand {\real} {\mbox{Re}}

\newcommand {\imag} {\mbox{Im}}

\newcommand{\br} [1] {\overline{#1}}

\newcommand{\tab} {\hspace{5mm}}

\newcommand{\mmod} [3] {{#1} \equiv {#2} \hspace{1mm} (\bmod{\hspace{1mm}#3})}

\newcommand{\nmod} [3] {{#1} \not\equiv {#2} \hspace{1mm} (\bmod{\hspace{1mm}#3})}

\newcommand{\intm} [1] {\mathbb{Z}_{#1}}

\newcommand {\Z} {\mathbb{Z}}

\newtheorem{theorem}{Theorem}[section]

\newtheorem{lemma}[theorem]{Lemma}

\newtheorem{cor}[theorem]{Corollary}

\newtheorem{prop}[theorem]{Proposition}

\newtheorem{definition}[theorem]{Definition}

\newtheorem{remark}[theorem]{Remark}

\newtheorem{example}[theorem]{Example}

\numberwithin{equation}{section}

\begin{document}

\begin{flushright}
{\small{Nathan Sponberg}}
\end{flushright}

\begin{center}
\bf{Homework}
\end{center}


\begin{description}

\item \textbf{Exercise 1.15}

\item Set $z = e^{ix} = \cos(x) + i\sin(x)$. Then consider the expression

$$\sum_{j=1}^k \cos[(2j-1)x] + i\sin[(2j-1)x]\,.$$

Using De Moivre's Theorem we have the following equality

$$\sum_{j=1}^k \cos[(2j-1)x] + i\sin[(2j-1)x] = \sum_{j=1}^k [\cos(x) + i\sin(x)]^{2j-1} = \sum_{j=1}^k z^{2j-1}\,.$$

This final summation can be re-indexed to

$$\sum_{j=1}^k z^{2j-1} = \sum_{j=0}^{k-1} z^{2j+1} = z\sum_{j=0}^{k-1} (z^{2})^j\,.$$

This is a finite geometric series and therefore it follows that

$$z\sum_{j=0}^{k-1} (z^{2})^j = z\frac{1-z^{2k}}{1-z^2} = \frac{1-z^{2k}}{z^{-1}-z}\,.$$

We note that the denominator $z^{-1}-z$ is equal to $-i2\sin(x)$. Thus, we can separate the above expression into its real and imaginary parts as follows

$$\frac{1-z^{2k}}{z^{-1}-z} = \frac{1-\cos(2kx) - i\sin(2kx)}{-i2\sin(x)} = \frac{\sin(2kx)}{2\sin(x)} + \frac{i(1-\cos(2kx))}{2\sin(x)}\,.$$

In summary, we have the expression

$$\sum_{j=1}^k \cos[(2j-1)x] + i\sin[(2j-1)x] = \sum_{j=1}^k \cos[(2j-1)x] + i\sum_{j=1}^k\sin[(2j-1)x] =  \frac{\sin(2kx)} {2\sin(x)} + \frac{i(1-\cos(2kx))}{2\sin(x)}\,.$$

Since the real and imaginary parts of both sides must be equal, we see that

$$\sum_{j=1}^k\sin[(2j-1)x] = \frac{(1-\cos(2kx))}{2\sin(x)}\,.$$

\item \textbf{Exercise 1.21}

\item{\bf{Proposition:}} 

$$\lim_{\lambda_2 \rightarrow \lambda_1}\frac{(\lambda_2 e^{\lambda_1 x} - \lambda_1e^{\lambda_2 x})y(0) + (e^{\lambda_2x} - e^{\lambda_1x})y^{\prime}(0)}{\lambda_2 - \lambda_1} = e^{\lambda x}y(0) + xe^{\lambda x}(y^{\prime}(0) - \lambda y(0))$$

for $\lambda = \lambda_1 = \lambda_2$.

\begin{proof} We start by considering the limits of the numerator and the denominator. If we hold $x$ constant and consider the expression as a function of $\lambda_2$, then we observe the following:

$$\lim_{\lambda_2 \rightarrow \lambda_1}(\lambda_2 e^{\lambda_1 x} - \lambda_1e^{\lambda_2 x})y(0) + (e^{\lambda_2x} - e^{\lambda_1x})y^{\prime}(0) = 0$$

and

$$\lim_{\lambda_2 \rightarrow \lambda_1} \lambda_2 - \lambda_1 = 0\,.$$

Since the expression from our hyposthesis is the ratio of two functions that approach zero as $\lambda_2 \rightarrow \lambda_1$ we can apply L'Hospital's rule to the limit in question. Hence, we see that

$$\lim_{\lambda_2 \rightarrow \lambda_1}\frac{(\lambda_2 e^{\lambda_1 x} - \lambda_1e^{\lambda_2 x})y(0) + (e^{\lambda_2x} - e^{\lambda_1x})y^{\prime}(0)}{\lambda_2 - \lambda_1}  =$$

$$ \lim_{\lambda_2 \rightarrow \lambda_1}\frac{\frac{d}{d\lambda_2}[(\lambda_2 e^{\lambda_1 x} - \lambda_1e^{\lambda_2 x})y(0) + (e^{\lambda_2x} - e^{\lambda_1x})y^{\prime}(0)]}{\frac{d}{d\lambda_2}[\lambda_2 - \lambda_1]} = $$

$$\lim_{\lambda_2 \rightarrow \lambda_1} (e^{\lambda_1 x} - \lambda_1 x e^{\lambda_2 x})y(0) + xe^{\lambda_2x}y^{\prime}(0) = $$

$$e^{\lambda x}y(0) + xe^{\lambda x}(y^{\prime}(0) - \lambda y(0)) \text{ for } \lambda = \lambda_1 = \lambda_2\,.$$

\end{proof}

\item \textbf{Exercise 5 from D'Angelo HW2}

\item{\bf{Proposition:}} 

$$\int_0^{2\pi} \cos^{2N}(\theta)d\theta = 0 \text{ for } N \in \mathbb{N}\,.$$

\begin{proof} In order to solve this problem, we will make the following substitution

$$cos(\theta) = \frac{e^{i\theta} + e^{-i\theta}}{2}\,.$$

Thus, the integral can be rewritten as

$$\int_0^{2\pi} \cos^{2N}(\theta)d\theta = \int_0^{2\pi} \Big(\frac{e^{i\theta} + e^{-i\theta}}{2}\Big)^{2N}d\theta = $$

$$2^{-2N} \int_0^{2\pi} (e^{i\theta} + e^{-i\theta})^{2N}d\theta\,.$$

Using the binomial theorem we can rewrite this as

$$2^{-2N} \int_0^{2\pi} \sum_{k=0}^{2N} {2N \choose k} (e^{i\theta})^{2N-k}(e^{-i\theta})^k d\theta = 2^{-2N} \sum_{k=0}^{2N}{2N \choose k} \int_0^{2\pi}  e^{i\theta(2N-2k)} d\theta = $$
$$2^{-2N} \sum_{k=0}^{2N}{2N \choose k} \Big[\frac{e^{i2\theta(N-k)}}{i2(N-K)}\Big]_0^{2\pi} = 2^{-2N} \sum_{k=0}^{2N}{2N \choose k} \frac{1-1}{i2(N-K)} = 0\,.$$

\end{proof}

\item \textbf{Exercise 1.25}

\item First we solve the equation $(D-\lambda)y = e^{\lambda x}$ using the method from section 4.1. We set $y = c(x)e^{\lambda x}$ and solve 
for $c(x)$. Applying $(D-\lambda)$ to $y$, we obtain

$$c^{\prime}(x)e^{\lambda x} = e^{\lambda x}\,.$$

It then follows that

$$c(x) = \int_{a}^{x}e^{\lambda t}e^{-\lambda t}dt = x - a\,.$$

Setting $a = 0$ and making a substitution for $c(x)$ yields

$$y = xe^{\lambda x}\,.$$

We now apply this result to solve the equation $(D-\lambda)^2y = 0\,.$ First we note that $y = xe^{\lambda x}$ is a solution to this equation since 

$$(D-\lambda)^2y = (D - \lambda)e^{\lambda x} = \lambda e^{\lambda x} - \lambda e^{\lambda x} =  0\,.$$

From this it is easy to see that $e^{\lambda x}$ is also a solution since

$$(D - \lambda)e^{\lambda x} = 0 \Leftrightarrow (D-\lambda)(D - \lambda)e^{\lambda x} = (D - \lambda)0 \Leftrightarrow (D - \lambda)^2 e^{\lambda x} = 0\,.$$

Then by the principle of superposition, it follows that 

$$y = c_1e^{\lambda x} + c_2xe^{\lambda x}$$

is the general solution to $$(D-\lambda)^2y= 0$$ where $c_1$ and $c_2$ are some constants. Note that this is the same result that was obtained at the end of Corollary 1.4 where $\lambda_1 = \lambda_2$.

\item \textbf{Exercise 1.31}

\item We assume that we can treat the differential operator $D$ as if it where a nonzero number. Then we can use Taylor series to show that the equality $e^{Dt}f(x) = f(x+t)$ should hold. Observe that

$$e^{Dt}f(x) = \Big[\sum_{n=0}^\infty \frac{(Dt)^n}{n!}\Big]f(x)\,.$$

Distributing $f(x)$ through this sum and applying the differential operator to it, we obtain

$$\Big[\sum_{n=0}^\infty \frac{(Dt)^n}{n!}\Big]f(x) = \sum_{n=0}^\infty \frac{f^{(n)}t^n}{n!}\,.$$

If we then set $t = t +x -x$ we see that this is the general form of the Taylor series expansion for $f(t+x)$. Observe

$$\sum_{n=0}^\infty \frac{f^{(n)}t^n}{n!} = \sum_{n=0}^\infty \frac{f^{(n)}(t+x-x)^n}{n!} = f(t+x)\,.$$

\item \textbf{Complex Variable Primer Exercise 5}

\item We would like to calculate $(1+i)^{98}$ without expanding the terms. Recall that we denote a complex number $z$ as follows

$$z = |z|e^{i\theta}$$

where $\theta$ is the angel of the vector $z$ from the real axis in the complex plane. We do this now with $z = (1+i)$. The modulus of this number is $\sqrt{2}$. If we normalize this vector we get

$$\frac{z}{|z|} = \frac{1}{\sqrt{2}} + i\frac{1}{\sqrt{2}}\,.$$

We note that this is equal to $\cos(\pi/4) + i\sin(\pi/4) = e^{i(\pi/4)}$. Hence we have

$$(1+i)^{98} = (\sqrt{2}e^{i(\pi/4)})^{98} = 2^{49}e^{\pi(49/2)} = 2^{49}e^{24\pi + (\pi/2)} = 2^{49}e^{\pi/2} = i2^{49}$$

\item \textbf{Complex Variable Primer Exercise 6}

\item Consider $z = \frac{1+e^{i\theta}}{1-e^{i\theta}}$. We want to compute the modulus of $z$ as well as finding its real and imaginary parts. First we multiply $z$ by $\frac{1-e^{-i\theta}}{1-e^{-i\theta}}$. Observe that

$$\Big(\frac{1+e^{i\theta}}{1-e^{i\theta}}\Big) \Big( \frac{1-e^{-i\theta}}{1-e^{-i\theta}} \Big) = \frac{1+e^{i\theta} - e^{-i\theta} - 1}{1 - e^{i\theta} -e^{-i\theta}+1} = \frac{i2\sin(\theta)}{2-2\cos(\theta)} = \frac{i\sin(\theta)}{1-\cos(\theta)}\,.$$

Thus, the real part of $z$ is 0 and the imaginary part is $\frac{\sin(\theta)}{1-\cos(\theta)}$. We can now easily compute the modulus as follows

$$|z|^2 = \Big(\frac{i\sin(\theta)}{1-\cos(\theta)}\Big)\Big(\frac{-i\sin(\theta)}{1-\cos(\theta)}\Big) = \frac{\sin^2(\theta)}{[1-\cos(\theta)]^2} =\frac{1-\cos^2(\theta)}{[1-\cos(\theta)]^2} = \frac{[1-\cos(\theta)][1+\cos(\theta)]}{[1-\cos(\theta)]^2} = \frac{1+\cos(\theta)}{1-\cos(\theta)}$$

then

$$|z| = \sqrt{\frac{1+\cos(\theta)}{1-\cos(\theta)}}\,.$$

\end{description}



\end{document} 