\documentclass[a4paper]{article}

\usepackage[a4paper,vmargin={20mm,20mm},hmargin={20mm,20mm}]{geometry}

\usepackage[pdftex]{graphicx}

\usepackage{amssymb, amsmath, amsthm}

\usepackage{enumitem}

\usepackage{tikz}

\usepackage{tkz-graph}

\newcommand {\C} [1] {{\mathbb C}^{#1}}

\newcommand {\R} [1] {{\mathbb R}^{#1}}

\newcommand {\limit} [2] {\displaystyle{\lim_{{#1}\rightarrow{#2}}}}

\newcommand {\bfrac} [2] {\displaystyle{\frac{#1}{#2}}}

\newcommand {\real} {\mbox{Re}}

\newcommand {\imag} {\mbox{Im}}

\newcommand{\br} [1] {\overline{#1}}

\newcommand{\tab} {\hspace{5mm}}

\newcommand{\mmod} [3] {{#1} \equiv {#2} \hspace{1mm} (\bmod{\hspace{1mm}#3})}

\newcommand{\nmod} [3] {{#1} \not\equiv {#2} \hspace{1mm} (\bmod{\hspace{1mm}#3})}

\newcommand{\intm} [1] {\mathbb{Z}_{#1}}

\newcommand {\Z} {\mathbb{Z}}

\newcommand {\threematrix} [9] {\small{\begin{bmatrix}{#1} & {#2} & {#3}\\{#4} & {#5} & {#6}\\{#7} & {#8} & {#9}\\\end{bmatrix}}}

\newcommand {\m} {\cdot}

\newtheorem{theorem}{Theorem}[section]

\newtheorem{lemma}[theorem]{Lemma}

\newtheorem{cor}[theorem]{Corollary}

\newtheorem{prop}[theorem]{Proposition}

\newtheorem{definition}[theorem]{Definition}

\newtheorem{remark}[theorem]{Remark}

\newtheorem{example}[theorem]{Example}

\numberwithin{equation}{section}

\begin{document}

\begin{flushright}
{\small{Nathan Sponberg\\}}
{\small{Math 564}}
\end{flushright}

\begin{center}
\bf{Homework 11}
\end{center}

\begin{description}

\item \textbf{Exercise 1.6}

\item \textbf{Proposition.} The series $\sum \limits_{n=0}^\infty\frac{\cos(kx)}{\log(k+2)}$ converges to a non-negative function.

\item\begin{proof} As the first step we will use summation by parts twice on the $N^{th}$ partial sum of the given series. Observe,

$$\sum \limits_{n=0}^N\frac{\cos(kx)}{\log(k+2)} = \frac{1}{\log(N+2)}\sum \limits_{n=0}^N\cos(nx) - \sum \limits_{n=0}^{N-1}\left( \frac{1}{\log(n+3)} -\frac{1}{\log(n+2)} \right)\sum \limits_{k=0}^{n}\cos(kx)\,.$$

Considering the right-hand term of this expression and summing it by parts again we have

$$- \sum \limits_{n=0}^{N-1}\left( \frac{1}{\log(n+3)} -\frac{1}{\log(n+2)} \right)\sum \limits_{k=0}^{n}\cos(kx) =$$

$$-\left( \frac{1}{\log(N+2)} -\frac{1}{\log(N+1)} \right)\sum \limits_{n=0}^{N-1}\sum \limits_{k=0}^{n}\cos(kx)$$
$$+ \sum \limits_{n=0}^{N-2}\left( \frac{1}{\log(n+4)} -\frac{2}{\log(n+3)} +\frac{1}{\log(n+2)} \right)\sum \limits_{k=0}^{n}\sum \limits_{j=0}^{k}\cos(jx)\,.$$

Thus, the final expression is

$$\frac{1}{\log(N+2)}\sum \limits_{n=0}^N\cos(nx)-\left( \frac{1}{\log(N+2)} -\frac{1}{\log(N+1)} \right)\sum \limits_{n=0}^{N-1}\sum \limits_{k=0}^{n}\cos(kx)$$
$$+ \sum \limits_{n=0}^{N-2}\left( \frac{1}{\log(n+4)} -\frac{2}{\log(n+3)} +\frac{1}{\log(n+2)} \right)\sum \limits_{k=0}^{n}\sum \limits_{j=0}^{k}\cos(jx)\,.$$\begin{flushright}
(*)
\end{flushright}

Before we consider the limit of this expression we will derive a formula for $\sum_{n=0}^{N}\sum_{k=0}^{n}\cos(kx)$. Consider the following

$$\sum \limits_{n=0}^{N}\sum \limits_{k=0}^{n}\cos(kx) = \sum \limits_{n=0}^{N}\sum \limits_{k=0}^{n}\frac{e^{inx} + e^{-inx}}{2} = \sum \limits_{n=0}^{N}\left(\frac{1}{2}+\frac{1}{2}\sum \limits_{-n}^{n}e^{inx}\right) = \frac{N}{2} + \frac{1}{2}\sum \limits_{n=0}^{N}\sum \limits_{-n}^{n}e^{inx} =$$
$$\frac{N+1}{2} + \frac{1}{2}\sum \limits_{n=1}^{N}\sum \limits_{-n}^{n}e^{inx}\,.$$

By the formula calculated in exercise 1.44 we observe that

$$\frac{N+1}{2} + \frac{1}{2}\sum \limits_{n=1}^{N}\sum \limits_{-n}^{n}e^{inx} = \frac{N+1}{2} + \frac{1}{2}\frac{\sin^2(\frac{Nx}{2})}{\sin^2(\frac{x}{2})}\,.$$

Making appropriate substitutions into eq.(*) we obtain

$$\frac{1}{\log(N+2)}\sum \limits_{n=0}^N\cos(nx)+\left( \frac{1}{\log(N+1)} -\frac{1}{\log(N+2)} \right)\left(\frac{N}{2} + \frac{1}{2}\frac{\sin^2(\frac{(N-1)x}{2})}{\sin^2(\frac{x}{2})}\right)$$
$$+ \sum \limits_{n=0}^{N-2}\left( \frac{1}{\log(n+4)} -\frac{2}{\log(n+3)} +\frac{1}{\log(n+2)} \right)\left(\frac{n+1}{2} + \frac{1}{2}\frac{\sin^2(\frac{nx}{2})}{\sin^2(\frac{x}{2})}\right)\,.$$

Now we take the limit of this sum as $N \rightarrow \infty$. We will consider each of the three summands separately. First observe that since $\sum_{n=0}^N\cos(nx)$ is bounded and $\log(N+2)$ is an increasing function it follows that

$$\lim\limits_{N \rightarrow \infty}\frac{1}{\log(N+2)}\sum \limits_{n=0}^N\cos(nx) = 0\,.$$

Next, we consider the limit of the second summand

$$\lim\limits_{N \rightarrow \infty}\left( \frac{1}{\log(N+1)} -\frac{1}{\log(N+2)} \right)\left(\frac{N}{2} + \frac{1}{2}\frac{\sin^2(\frac{(N-1)x}{2})}{\sin^2(\frac{x}{2})}\right) = $$
$$\lim\limits_{N \rightarrow \infty}\left( \frac{N}{2\log(N+1)} -\frac{N}{2\log(N+2)} \right)+\lim\limits_{N \rightarrow \infty}\left( \frac{1}{\log(N+1)} -\frac{1}{\log(N+2)} \right) \frac{1}{2}\frac{\sin^2(\frac{(N-1)x}{2})}{\sin^2(\frac{x}{2})}\,.$$

Note that first limit goes to zero (\textit{not sure how to show this but seems really similar to exercise 1.47}), which leaves us with

$$\lim\limits_{N \rightarrow \infty}\left( \frac{1}{\log(N+1)} -\frac{1}{\log(N+2)} \right) \frac{1}{2}\frac{\sin^2(\frac{(N-1)x}{2})}{\sin^2(\frac{x}{2})}\,.$$

Note that for each fixed value of $x$ $\frac{\sin^2((N-1)x/2)}{\sin^2(x/2)}$ is bounded. Therefore this second limit also goes to zero. Finally, we consider the limit of the last summand

$$\lim\limits_{N \rightarrow \infty}\sum \limits_{n=0}^{N-2}\left( \frac{1}{\log(n+4)} -\frac{2}{\log(n+3)} +\frac{1}{\log(n+2)} \right)\left(\frac{n+1}{2} + \frac{1}{2}\frac{\sin^2(\frac{nx}{2})}{\sin^2(\frac{x}{2})}\right)\,.$$

\textit{I don't know how to proceed from here.}

\end{proof}

\item \textbf{Exercise 3.1}

\item \textbf{Proposition.} The function $f(x) = e^{-x^2}$ is in the Schwartz space.

\begin{proof} First we will show that $e^{-x^2}$ is smooth. First we calculate the first few derivatives of $f$, observe that

$$f^{\prime} = -2xe^{-x^2}$$

$$f^{\prime\prime} = (-2 +4x^2)e^{-x^2}$$

and

$$f^{\prime\prime\prime} = (12x-8x^3)e^{-x^2}\,.$$

Thus, we see that the pattern seems to suggest that each $n^{th}$ derivative will be equal to a polynomial of degree $n$ times $e^{-x^2}$. We prove this using induction on $n$. Suppose that for $n$ we have

$$f^{(n)} = P_ne^{-x^2}\,,$$

where $P_n$ is a polynomial in $x$ of degree $n$. Then observe that by the product rule of differentiation

$$f^{(n+1)} = P_n^{\prime}e^{-x^2} + P_n(-2xe^{-x^2}) = (P_n^{\prime} -2xP_n)e^{-x^2}\,.$$ 

Since $(P_n^{\prime} -2xP_n)$ is a polynomial of degree $n+1$, it follows that the result holds for all $n \in \mathbb{N}$, by induction on $n$.

Next we show that $f$ is indeed in the Schwartz space. We must show for all positive $a,n \in \mathbb{Z}$ that 

$$\lim \limits_{|x|\rightarrow \infty} = |x|^a(\frac{d}{dx})^n f(x) = 0\,.$$

Since $f(x)$ is an even function, we will consider only non-negative values of $x$ and and the limit at positive infinity. The limit at negative infinity is symmetric. We have previously established that the $n^{th}$ derivative of $f$ is the product of a $n^{th}$ degree polynomial, $P_n$, and $e^{-x^2}$. Thus, in general

$$x^a(\frac{d}{dx})^n f(x) = \frac{x^a(P_n)}{e^{x^2}}\,.$$

Here, $x^a(P_n)$ is an $(a+n)^{th}$ degree polynomial. Since the limit of the numerator and the denominator are both infinity as $x \rightarrow \infty$, we can apply L'Hospital’s rule repeatedly (since $x^a(P_n)$ and $e^{x^2}$ are both smooth). Hence, taking the $(a+n)^{th}$ derivative of both the numerator and the denominator, we see that

$$\frac{(\frac{d}{dx})^{a+n}x^a(P_n)}{(\frac{d}{dx})^{a+n}e^{x^2}} = \frac{c}{P_{a+n}e^{x^2}}$$

where $P_{a+n}$ is a $(a+n)^{th}$ degree polynomial and $c$ is a constant. It is easy to see that the limit of this ratio as $x \rightarrow \infty$ is equal to zero. Thus, $f(x) = e^{-x^2}$ is in the Schwartz space.

\end{proof}



\end{description}

\end{document} 