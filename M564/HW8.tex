\documentclass[a4paper]{article}

\usepackage[a4paper,vmargin={20mm,20mm},hmargin={20mm,20mm}]{geometry}

\usepackage[pdftex]{graphicx}

\usepackage{amssymb, amsmath, amsthm}

\usepackage{enumitem}

\usepackage{tikz}

\usepackage{tkz-graph}

\newcommand {\C} [1] {{\mathbb C}^{#1}}

\newcommand {\R} [1] {{\mathbb R}^{#1}}

\newcommand {\limit} [2] {\displaystyle{\lim_{{#1}\rightarrow{#2}}}}

\newcommand {\bfrac} [2] {\displaystyle{\frac{#1}{#2}}}

\newcommand {\real} {\mbox{Re}}

\newcommand {\imag} {\mbox{Im}}

\newcommand{\br} [1] {\overline{#1}}

\newcommand{\tab} {\hspace{5mm}}

\newcommand{\mmod} [3] {{#1} \equiv {#2} \hspace{1mm} (\bmod{\hspace{1mm}#3})}

\newcommand{\nmod} [3] {{#1} \not\equiv {#2} \hspace{1mm} (\bmod{\hspace{1mm}#3})}

\newcommand{\intm} [1] {\mathbb{Z}_{#1}}

\newcommand {\Z} {\mathbb{Z}}

\newcommand {\threematrix} [9] {\small{\begin{bmatrix}{#1} & {#2} & {#3}\\{#4} & {#5} & {#6}\\{#7} & {#8} & {#9}\\\end{bmatrix}}}

\newcommand {\m} {\cdot}

\newtheorem{theorem}{Theorem}[section]

\newtheorem{lemma}[theorem]{Lemma}

\newtheorem{cor}[theorem]{Corollary}

\newtheorem{prop}[theorem]{Proposition}

\newtheorem{definition}[theorem]{Definition}

\newtheorem{remark}[theorem]{Remark}

\newtheorem{example}[theorem]{Example}

\numberwithin{equation}{section}

\begin{document}

\begin{flushright}
{\small{Nathan Sponberg\\}}
{\small{Math 564}}
\end{flushright}

\begin{center}
\bf{Homework 7}
\end{center}

\begin{description}

\item \textbf{Exercise 2.11}

\item \textbf{Proposition.} Given a projection $P$ on a Hilbert space $\mathcal{H}$, the following holds:

\begin{enumerate}

\item $I-P$ is also a projection
\item $\mathcal{R}(P) = \mathcal{N}(I-P)$
\item $\mathcal{H} = \mathcal{R}(P) + \mathcal{N}(P)$

\end{enumerate}

\begin{proof} Let $z \in \mathcal{H}$, then observe that

$$[I(z) - P(z)]^2 = I(I(z)) -2I(P(z)) + P(P(z)) = I(z) - 2P(z) + P(z) = I(z) - P(z)\,.$$

Thus the first result holds. 

Next we will show that $\mathcal{R}(P) \subseteq \mathcal{N}(I-P)$ and $\mathcal{R}(P) \supseteq \mathcal{N}(I-P)$. First observe that given $z \in \mathcal{R}(P)$, there must exist some $w \in \mathcal{H}$ such that $P(w) = z$. However since $P$ is a projection on $\mathcal{H}$, it must also hold that $P(P(w)) = P(z) = z$. Thus $z$ is also the image under $P$ of itself. It follows that $(I-P)(z) = z-z = 0$ and therefore $z \in \mathcal{N}(I-P)$. Now given $z \in \mathcal{N}(I-P)$ we observe that $I(z) - P(z) = z - P(z) = 0$, which implies $P(z) = z$. Hence $z \in \mathcal{R}(P)$. Consequently, we see that result two holds as well.

For the final result, it is clear that $\mathcal{R}(P)+\mathcal{N}(P) \subseteq \mathcal{H}$. Therefore we will show that $\mathcal{H} \subseteq \mathcal{R}(P)+\mathcal{N}(P)$. 

\textit{I could not figure out how to make this work. Am I not understanding what is being asked? Is it not the union of the two sets?}

\end{proof}

\item \textbf{Exercise 2.12}

\item \textbf{Proposition.} For a fixed $w$ in a Hilbert space $\mathcal{H}$, linear operator $P(v)$ on $\mathcal{H}$ given by 

$$P(v) = \frac{\langle v,w \rangle}{\|w\|^2}w$$

is a projection on $\mathcal{H}$.

\begin{proof} Observe that 

$$P(P(v)) = P\Big(\frac{\langle v,w \rangle}{\|w\|^2}w\Big) = \frac{\langle \frac{\langle v,w \rangle}{\|w\|^2}w,w \rangle}{\|w\|^2}w\,.$$

Since the inner product of a Hilbert space $\mathcal{H}$ is a complex scalar we can factor out $\langle v,w \rangle/ \|w\|^2$. Thus, we see that

$$\frac{\langle \frac{\langle v,w \rangle}{\|w\|^2}w,w \rangle}{\|w\|^2}w = \frac{\langle v,w \rangle}{\|w\|^2}\frac{\langle w,w \rangle}{\|w\|^2}w = \frac{\langle v,w \rangle}{\|w\|^2}w = P(v)\,.$$

\end{proof}

\item \textbf{Exercise 2.13}

\item \textbf{Proposition.} Let $\mathcal{H} = L^2[-1,1]$ and $V_e$ and $V_o$ be the subspaces of even and odd functions, respectively, in $\mathcal{H}$. Then $V_e$ is orthogonal to $V_o$.

\begin{proof} The inner product in $L^2[-1,1]$ is defined as

$$\langle f,g \rangle = \int_{-1}^1 f(x)g(x)dx\,.$$

If $f$ is an arbitrary function in $V_e$ and $g$ is an arbitrary function in $V_o$, then we observe that $fg$ must itself be an odd function. Hence, it follows that for all $f \in V_e$ and all $g \in V_o$

$$\langle f,g \rangle = \int_{-1}^1 f(x)g(x)dx = 0\,.$$
 
Therefore, $V_e$ and $V_o$ are orthogonal subspaces of $\mathcal{H}$.

\end{proof}

\item \textbf{Exercise 2.16}

\item First note that the inequality 
$$0 \leq \|z - \sum_{n=1}^N\langle z,z_n \rangle z_n\|^2$$

must be true since the squared norm in a Hilbert space will always return a non-negative real number. Next, observe that 

$$\|z - \sum_{n=1}^N\langle z,z_n \rangle z_n\|^2 = \Big\langle z-\sum_{n=1}^N\langle z,z_n \rangle z_n, z -\sum_{n=1}^N\langle z,z_n \rangle z_n \Big\rangle = $$

$$\Big\langle z,z \Big\rangle - \Big\langle z,\sum_{n=1}^N\langle z,z_n \rangle z_n \Big\rangle - \Big\langle \sum_{n=1}^N\langle z,z_n \rangle z_n,z \Big\rangle + \Big\langle \sum_{n=1}^N\langle z,z_n \rangle z_n,\sum_{n=1}^N\langle z,z_n \rangle z_n \Big\rangle = $$

$$\|z\|^2 - \sum_{n=1}^N \overline{\langle z,z_n \rangle}\langle z, z_n \rangle - \sum_{n=1}^N\langle z,z_n \rangle \overline{\langle  z,z_n \rangle} + \|\sum_{n=1}^N\langle z,z_n \rangle z_n\|^2 = $$

$$\|z\|^2 - 2\sum_{n=1}^N |\langle z,z_n \rangle|^2 + \|z_n\|^2|\sum_{n=1}^N\langle z,z_n \rangle|^2 = \|z\|^2 - \sum_{n=1}^N |\langle z,z_n \rangle|^2\,.$$

Thus, we have 

$$0 \leq \|z - \sum_{n=1}^N\langle z,z_n \rangle z_n\|^2 = \|z\|^2 - \sum_{n=1}^N |\langle z,z_n \rangle|^2\,.$$

\item \textbf{Exercise 2.17}

\item \textbf{Proposition.} For $\mathcal{H} = L^2[0,1]$, the orthogonal projection of $x^2$ onto span$(\lbrace 1,x \rbrace)$ is $x-1/6$. For $\mathcal{H} = L^2[-1,1]$, the orthogonal projection of $x^2$ onto span$(\lbrace 1,x \rbrace)$ is $1/3$.

\begin{proof} We will find the minimum of the expression $\|x^2 - xa - b\|$ which will give us the orthogonal projection onto the spanning set. First we consider the case of $L^2[0,1]$. Observe that

$$\|x^2 - xa - b\| = \int_0^1 (x^2 - xa - b)^2 = \int_0^1 x^4 -2ax^3 -2bx^2+a^2x^2+2abx+b^2 dx = $$

$$\frac{1}{5} - \frac{1}{2}a - \frac{2}{3}b + \frac{1}{3}a^2 + ab + b^2\,.$$

We then take partial derivatives of this function in terms of $a$ and $b$. This yields

$$\frac{\partial}{\partial a} = - \frac{1}{2} + \frac{2}{3}a + b$$

and

$$\frac{\partial}{\partial b} = - \frac{2}{3} + a + 2b\,.$$

Setting these equal to zero we obtain

$$b =  \frac{1}{2} -\frac{2}{3}a$$

and

$$ a =  \frac{2}{3} - 2b\,.$$

From this it follows that $a = 1$ and $b = -1/6$. Next we consider the case of $L^2[-1,1]$. Observe that

$$\|x^2 - xa - b\| = \int_{-1}^1 (x^2 - xa - b)^2 = \int_{-1}^1 x^4 -2ax^3 -2bx^2+a^2x^2+2abx+b^2 dx = $$

$$\frac{2}{5} - \frac{4}{3}b + \frac{2}{3}a^2 + 2b^2\,.$$

We then take partial derivatives of this function in terms of $a$ and $b$. This yields

$$\frac{\partial}{\partial a} = \frac{4}{3}a$$

and

$$\frac{\partial}{\partial b} = - \frac{4}{3} + 4b\,.$$

Setting these equal to zero we obtain

$$b =  \frac{1}{3}$$

and

$$ a =  0\,.$$

From this it follows that $a = 0$ and $b = 1/3$.

\end{proof}
e44
\end{description}

\end{document} 