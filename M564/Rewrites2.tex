\documentclass[a4paper]{article}

\usepackage[a4paper,vmargin={20mm,20mm},hmargin={20mm,20mm}]{geometry}

\usepackage[pdftex]{graphicx}

\usepackage{amssymb, amsmath, amsthm}

\usepackage{enumitem}

\usepackage{tikz}

\usepackage{tkz-graph}

\newcommand {\C} [1] {{\mathbb C}^{#1}}

\newcommand {\R} [1] {{\mathbb R}^{#1}}

\newcommand {\limit} [2] {\displaystyle{\lim_{{#1}\rightarrow{#2}}}}

\newcommand {\bfrac} [2] {\displaystyle{\frac{#1}{#2}}}

\newcommand {\real} {\mbox{Re}}

\newcommand {\imag} {\mbox{Im}}

\newcommand{\br} [1] {\overline{#1}}

\newcommand{\tab} {\hspace{5mm}}

\newcommand{\mmod} [3] {{#1} \equiv {#2} \hspace{1mm} (\bmod{\hspace{1mm}#3})}

\newcommand{\nmod} [3] {{#1} \not\equiv {#2} \hspace{1mm} (\bmod{\hspace{1mm}#3})}

\newcommand{\intm} [1] {\mathbb{Z}_{#1}}

\newcommand {\Z} {\mathbb{Z}}

\newcommand {\threematrix} [9] {\small{\begin{bmatrix}{#1} & {#2} & {#3}\\{#4} & {#5} & {#6}\\{#7} & {#8} & {#9}\\\end{bmatrix}}}

\newcommand {\m} {\cdot}

\newtheorem{theorem}{Theorem}[section]

\newtheorem{lemma}[theorem]{Lemma}

\newtheorem{cor}[theorem]{Corollary}

\newtheorem{prop}[theorem]{Proposition}

\newtheorem{definition}[theorem]{Definition}

\newtheorem{remark}[theorem]{Remark}

\newtheorem{example}[theorem]{Example}

\numberwithin{equation}{section}

\begin{document}

\begin{flushright}
{\small{Nathan Sponberg\\}}
{\small{Math 564}}
\end{flushright}

\begin{center}
\bf{Rewrites 2}
\end{center}

\begin{description}

\item \textbf{Exercise 1.5 (Original)} 

\textbf{Proposition.} Cauchy sequences of complex numbers converge if and only if, whenever a series $\sum |a_n|$ converges, then $\sum a_n$ converges as well.

\begin{proof} First we assume the convergence of Cauchy sequences. We will show that this implies that absolutely convergent series conditionally converge as well. Let ${a_n}$ be a sequence of complex numbers such that $\sum_{n=1}^\infty |a_n|$ converges. This implies that the sequence of partial sums of $\sum_{n=1}^\infty |a_n|$ (we denote the $N^th$ partial sum as $|A|_N$) converges and therefore must be a Cauchy sequence. In particular, this implies that given $\epsilon > 0$, there exists some $N$ such that for $K,M \geq N$ (without lose of generality we can assume that $K > M$) 

	$$||A|_K - |A|_M| < \epsilon\,.$$
	
	Observe that 
	$$\epsilon > ||A|_K - |A|_M| = |A|_K - |A|_M = \sum_{M+1}^K |a_n| \geq |\sum_{M+1}^K a_n| = |A_K +A_M|\,,$$
	
	where $A_M$ is the $M^{th}$ partial sum of $\sum_{n=1}^\infty a_n$. Thus, we see that ${A_N}$ is a Cauchy sequence as well and therefore it converges. 

Next assume that if a series converges absolutely then, it converges conditionally. We will show that this implies the convergence of Cauchy sequences. Let ${a_n}$ be a Cauchy sequence of complex numbers. Since ${a_n}$ is Cauchy there exists some $N_1$ such that for $n > N_1$

$$|a_n - a_{N_1}| < \frac{1}{2}\,.$$

Similarly, there exists some $N_2$ such that for $n > N_2$

$$|a_n - a_{N_2}| < \frac{1}{2^2}\,.$$

Proceeding inductively we see that in general We can find a $N_k$ such that for $n > N_k$

$$|a_n - a_{N_k}| < \frac{1}{2^k}\,.$$

Note that we have created an increasing sequence of indices $N_1 < N_2 < ... < N_k$. Thus we can replace some terms in the above inequalities as follows

	$$|a_n - a_{N_k}| < \frac{1}{2^k}$$
	$$|a_{N_k} - a_{N_{k-1}}| < \frac{1}{2^{k-1}}$$
	$$\m$$
	$$\m$$
	$$\m$$
	$$|a_{N_2} - a_{N_{1}}| < \frac{1}{2}\,,$$
	
	where $n > N_k$. Setting $n = N_{K+1}$ and  combining these inequalities into sums we obtain
	
	$$\sum_{k=1}^K |a_{N_{k+1}} - a_{N_{k}}| < \sum_{k=1}^K \frac{1}{2^k}\,.$$
	
	Observe that the right-hand side of this inequality is a geometric series that converges as $K \rightarrow \infty$. Therefore the left-hand side must also be a convergent series. By our hypothesis, since $\sum_{k=1}^K |a_{N_{k+1}} - a_{N_{k}}|$ is convergent, it follows that $\sum_{k=1}^K a_{N_{k+1}} - a_{N_{k}}$ is convergent as well.  Observe that this series is telescopic and therefore
	
	$$\sum_{k=1}^K a_{N_{k+1}} - a_{N_{k}} = a_{N_{k+1}} = a_n$$
	
	for $n > N_{k}$. Hence, the Cauchy sequence ${a_n}$ is convergent.

\end{proof}

\item \textbf{Rewrite of Exercise 1.5} 

\textbf{Proposition.} Cauchy sequences of complex numbers converge if and only if, whenever a series $\sum |a_n|$ converges, then $\sum a_n$ converges as well.

\begin{proof} First we assume the convergence of Cauchy sequences. We will show that this implies that absolutely convergent series conditionally converge as well. Let $\lbrace a_n \rbrace$ be a sequence of complex numbers such that $\sum_{n=1}^\infty |a_n|$ converges. This implies that the sequence of partial sums of $\sum_{n=1}^\infty |a_n|$ (we denote the $N^th$ partial sum as $|A|_N$) converges and therefore must be a Cauchy sequence. In particular, this implies that given $\epsilon > 0$, there exists some $N$ such that for $K,M \geq N$ (without lose of generality we can assume that $K > M$) 

	$$||A|_K - |A|_M| < \epsilon\,.$$
	
	Observe that 
	$$\epsilon > ||A|_K - |A|_M| = |A|_K - |A|_M = \sum_{M+1}^K |a_n| \geq |\sum_{M+1}^K a_n| = |A_K +A_M|\,,$$
	
	where $A_M$ is the $M^{th}$ partial sum of $\sum_{n=1}^\infty a_n$. Thus, we see that ${A_N}$ is a Cauchy sequence as well and therefore it converges. 

Next assume that if a series converges absolutely then, it converges conditionally. We will show that this implies the convergence of Cauchy sequences. Let $\lbrace a_n \rbrace$ be a Cauchy sequence of complex numbers. Since $\lbrace a_n \rbrace$ is Cauchy there exists some $N_1$ such that for $n > N_1$

$$|a_n - a_{N_1}| < \frac{1}{2}\,.$$

Similarly, there exists some $N_2 > N_1$ such that for $n > N_2$

$$|a_n - a_{N_2}| < \frac{1}{2^2}\,.$$

Proceeding inductively we see that in general We can find a $N_k > N_{k-1}$ such that for $n > N_k$

$$|a_n - a_{N_k}| < \frac{1}{2^k}\,.$$

Note that we have created an increasing sequence of indices $N_1 < N_2 < ... < N_k$. Thus we can replace some terms in the above inequalities as follows

	$$|a_{N_{k+1}} - a_{N_k}| < \frac{1}{2^k}$$
	$$|a_{N_k} - a_{N_{k-1}}| < \frac{1}{2^{k-1}}$$
	$$\m$$
	$$\m$$
	$$\m$$
	$$|a_{N_2} - a_{N_{1}}| < \frac{1}{2}\,,$$
	
	Combining these inequalities into sums we obtain
	
	$$\sum_{k=1}^K |a_{N_{k+1}} - a_{N_{k}}| < \sum_{k=1}^K \frac{1}{2^k}\,.$$
	
	Observe that the right-hand side of this inequality is a geometric series that converges as $K \rightarrow \infty$. Therefore the left-hand side must also be a convergent series. By our hypothesis, since $\sum_{k=1}^K |a_{N_{k+1}} - a_{N_{k}}|$ is convergent, it follows that $\sum_{k=1}^K\left( a_{N_{k+1}} - a_{N_{k}}\right)$ is convergent as well.  Observe that this series is telescopic and therefore
	
	$$\sum_{k=1}^K a_{N_{k+1}} - a_{N_{k}} = a_{N_{K+1}} - a_{N_1}\,.$$
	
	Thus, we have shown that $\langle a_n \rangle$ contains a convergent subsequence. This means that the Cauchy sequence ${a_n}$ must be convergent as well.

\end{proof}


\item \textbf{Exercise 2.11 (Original)}

\item \textbf{Proposition.} Given a projection $P$ on a Hilbert space $\mathcal{H}$, the following holds:

\begin{enumerate}

\item $I-P$ is also a projection
\item $\mathcal{R}(P) = \mathcal{N}(I-P)$
\item $\mathcal{H} = \mathcal{R}(P) + \mathcal{N}(P)$

\end{enumerate}

\begin{proof} Let $z \in \mathcal{H}$, then observe that

$$[I(z) - P(z)]^2 = I(I(z)) -2I(P(z)) + P(P(z)) = I(z) - 2P(z) + P(z) = I(z) - P(z)\,.$$

Thus the first result holds. 

Next we will show that $\mathcal{R}(P) \subseteq \mathcal{N}(I-P)$ and $\mathcal{R}(P) \supseteq \mathcal{N}(I-P)$. First observe that given $z \in \mathcal{R}(P)$, there must exist some $w \in \mathcal{H}$ such that $P(w) = z$. However since $P$ is a projection on $\mathcal{H}$, it must also hold that $P(P(w)) = P(z) = z$. Thus $z$ is also the image under $P$ of itself. It follows that $(I-P)(z) = z-z = 0$ and therefore $z \in \mathcal{N}(I-P)$. Now given $z \in \mathcal{N}(I-P)$ we observe that $I(z) - P(z) = z - P(z) = 0$, which implies $P(z) = z$. Hence $z \in \mathcal{R}(P)$. Consequently, we see that result two holds as well.

For the final result, it is clear that $\mathcal{R}(P)+\mathcal{N}(P) \subseteq \mathcal{H}$. Therefore we will show that $\mathcal{H} \subseteq \mathcal{R}(P)+\mathcal{N}(P)$. 

\textit{I could not figure out how to make this work. Am I not understanding what is being asked? Is it not the union of the two sets?}

\end{proof}


\item \textbf{Rewrite of Exercise 2.11}

\item \textbf{Proposition.} Given a projection $P$ on a Hilbert space $\mathcal{H}$, the following holds:

\begin{enumerate}

\item $I-P$ is also a projection
\item $\mathcal{R}(P) = \mathcal{N}(I-P)$
\item $\mathcal{H} = \mathcal{R}(P) + \mathcal{N}(P)$

\end{enumerate}

\begin{proof} Let $z \in \mathcal{H}$, then observe that

$$[I(z) - P(z)]^2 = I(I(z)) -2I(P(z)) + P(P(z)) = I(z) - 2P(z) + P(z) = I(z) - P(z)\,.$$

Thus the first result holds. 

Next we will show that $\mathcal{R}(P) \subseteq \mathcal{N}(I-P)$ and $\mathcal{R}(P) \supseteq \mathcal{N}(I-P)$. First observe that given $z \in \mathcal{R}(P)$, there must exist some $w \in \mathcal{H}$ such that $P(w) = z$. However since $P$ is a projection on $\mathcal{H}$, it must also hold that $P(P(w)) = P(z) = z$. Thus $z$ is also the image under $P$ of itself. It follows that $(I-P)(z) = z-z = 0$ and therefore $z \in \mathcal{N}(I-P)$. Now given $z \in \mathcal{N}(I-P)$ we observe that $I(z) - P(z) = z - P(z) = 0$, which implies $P(z) = z$. Hence $z \in \mathcal{R}(P)$. Consequently, we see that result two holds as well.

For the final result, it is clear that $\mathcal{R}(P)+\mathcal{N}(P) \subseteq \mathcal{H}$. Therefore we will show that $\mathcal{H} \subseteq \mathcal{R}(P)+\mathcal{N}(P)$. Given $z \in \mathcal{H}$ observe that

$$z = P(z) + z - P(z) = P(z) + I(z) - P(z) = P(z) + (I-P)(z)\,.$$

Note that $P(z) \in \mathcal{R}(P)$ and $(I-P)(z) \in \mathcal{R}(I-P)$. From the previous results we know that

$$\mathcal{R}(I-P) = \mathcal{N}(P)\,.$$

Hence, we see that for all $z \in \mathcal{H}$, 

$$z \in \mathcal{R}(P) + \mathcal{N}(P)\,.$$

It follows that $\mathcal{H} = \mathcal{R}(P) + \mathcal{N}(P)$.

\end{proof}

\item \textbf{Exercise 5 from D'Angelo HW2 (Original)}

\item{\bf{Proposition:}} 

$$\int_0^{2\pi} \cos^{2N}(\theta)d\theta = 0 \text{ for } N \in \mathbb{N}\,.$$

\begin{proof} In order to solve this problem, we will make the following substitution

$$cos(\theta) = \frac{e^{i\theta} + e^{-i\theta}}{2}\,.$$

Thus, the integral can be rewritten as

$$\int_0^{2\pi} \cos^{2N}(\theta)d\theta = \int_0^{2\pi} \Big(\frac{e^{i\theta} + e^{-i\theta}}{2}\Big)^{2N}d\theta = $$

$$2^{-2N} \int_0^{2\pi} (e^{i\theta} + e^{-i\theta})^{2N}d\theta\,.$$

Using the binomial theorem we can rewrite this as

$$2^{-2N} \int_0^{2\pi} \sum_{k=0}^{2N} {2N \choose k} (e^{i\theta})^{2N-k}(e^{-i\theta})^k d\theta = 2^{-2N} \sum_{k=0}^{2N}{2N \choose k} \int_0^{2\pi}  e^{i\theta(2N-2k)} d\theta = $$
$$2^{-2N} \sum_{k=0}^{2N}{2N \choose k} \Big[\frac{e^{i2\theta(N-k)}}{i2(N-K)}\Big]_0^{2\pi} = 2^{-2N} \sum_{k=0}^{2N}{2N \choose k} \frac{1-1}{i2(N-K)} = 0\,.$$

\end{proof}

\item \textbf{Rewrite of Exercise 5 from D'Angelo HW2}

\item{\bf{Proposition:}} 

$$\int_0^{2\pi} \cos^{2N}(\theta)d\theta = 0 \text{ for } N \in \mathbb{N}\,.$$

\begin{proof} In order to solve this problem, we will make the following substitution

$$cos(\theta) = \frac{e^{i\theta} + e^{-i\theta}}{2}\,.$$

Thus, the integral can be rewritten as

$$\int_0^{2\pi} \cos^{2N}(\theta)d\theta = \int_0^{2\pi} \Big(\frac{e^{i\theta} + e^{-i\theta}}{2}\Big)^{2N}d\theta = $$

$$2^{-2N} \int_0^{2\pi} (e^{i\theta} + e^{-i\theta})^{2N}d\theta\,.$$

Using the binomial theorem we can rewrite this as

$$2^{-2N} \int_0^{2\pi} \sum_{k=0}^{2N} {2N \choose k} (e^{i\theta})^{2N-k}(e^{-i\theta})^k d\theta = 2^{-2N} \sum_{k=0}^{2N}{2N \choose k} \int_0^{2\pi}  e^{i\theta(2N-2k)} d\theta = $$
$$2^{-2N} \sum_{k=0,k\neq N}^{2N}{2N \choose k} \Big[\frac{e^{i2\theta(N-k)}}{i2(N-K)}\Big]_0^{2\pi} + {2N \choose N} \int_0^{2\pi}  e^{i\theta0} d\theta = $$
$$2^{-2N} \sum_{k=0}^{2N}{2N \choose k} \frac{1-1}{i2(N-K)} + {2N \choose N} \int_0^{2\pi} d\theta = 2\pi{2N \choose N}\,.$$

\end{proof}

\end{description}

\end{document} 